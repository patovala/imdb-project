%%%%%%%%%%%%%%%%%%%%%%%%%%%%%%%%%%%%%%%%%
% Simple Sectioned Essay Template
% LaTeX Template
%
% This template has been downloaded from:
% http://www.latextemplates.com
%
% Modified by : Ivan Patricio Valarezo (c) patovala@pupilabox.net.ec
%               The University of Melbourne - MSSE
%
%%%%%%%%%%%%%%%%%%%%%%%%%%%%%%%%%%%%%%%%%
% Why don't put the commands to compile this??
% PV: To compile the bibfile: bibtex COMP90050, then use the command \ll as usual 
%     from the vim environment
% 
%------------------------------------------------------------------------------------
%	PACKAGES AND OTHER DOCUMENT CONFIGURATIONS
%------------------------------------------------------------------------------------

\documentclass[10pt]{article} % Default font size is 12pt, it can be changed here

\usepackage{geometry} % Required to change the page size to A4
\geometry{a4paper} % Set the page size to be A4 as opposed to the default US Letter

\usepackage{graphicx} % Required for including pictures

\usepackage{float} % Allows putting an [H] in \begin{figure} to specify the exact location of the figure
\usepackage{wrapfig} % Allows in-line images such as the example fish picture

\usepackage{lipsum} % Used for inserting dummy 'Lorem ipsum' text into the template

\usepackage{url} % url package

\usepackage{apacite} % citations style apa 

\usepackage{longtable} % long tables breaking on each page 

\linespread{1.2} % Line spacing

%\setlength\parindent{0pt} % Uncomment to remove all indentation from paragraphs

\graphicspath{{./pictures/}} % Specifies the directory where pictures are stored

\begin{document}

%------------------------------------------------------------------------------------
%	TITLE PAGE
%------------------------------------------------------------------------------------

\begin{titlepage}

\newcommand{\HRule}{\rule{\linewidth}{0.5mm}} % Defines a new command for the horizontal lines, change thickness here

\center % Center everything on the page

\textsc{\LARGE University of Melbourne}\\[1.5cm] % Name of your university/college
\textsc{\Large Advanced Database Systems}\\[0.5cm] % Major heading such as course name
\textsc{\large COMP90050}\\[0.5cm] % Minor heading such as course title

\HRule \\[0.4cm]
{ \huge \bfseries In Memory Database Management Systems}\\[0.4cm] % Title of your document
\HRule \\[1.5cm]

\begin{minipage}{0.4\textwidth}
\begin{flushleft} \large
\emph{Authors:}\\
Hammad \\
Saman \\
Ivan Patricio \textsc{Valarezo}\\ % Your name
\end{flushleft}
\end{minipage}
~
\begin{minipage}{0.4\textwidth}
\begin{flushright} \large
%\emph{id:} \\ % I want to put my id here
ID: \textsc{601099} % 
\end{flushright}
\end{minipage}\\[4cm]

{\large \today}\\[3cm] % Date, change the \today to a set date if you want to be precise

%\includegraphics{Logo}\\[1cm] % Include a department/university logo - this will require the graphicx package

\vfill % Fill the rest of the page with whitespace

\end{titlepage}


%------------------------------------------------------------------------------------
%	MY PERSONAL REFERENCE SECTION
%------------------------------------------------------------------------------------
% Pato, put here all the things, tips, references, etc, that you think you'll use in 
% the future.
%
% how to cite according to the UNIMELB http://www.library.unimelb.edu.au/cite/


%------------------------------------------------------------------------------------
%	TABLE OF CONTENTS
%------------------------------------------------------------------------------------

\newpage % Begins the essay on a new page instead of on the same page as the table of contents 

%------------------------------------------------------------------------------------
%	INTRODUCTION
%------------------------------------------------------------------------------------
\begin{abstract}

Here we need to put the abstract of the work done.
 
\end{abstract}


\section{Introduction} 

The main Introduction.

%------------------------------------------------------------------------------------
%	Background and Analysis <-- TODO: is the name for this section accurate? 
%------------------------------------------------------------------------------------
% Let's talk about the main topic in terms of three major parts:
% - Data Organization
% - Query Processing
% - Data Recovery

\section{Background and Analysis} 
\subsection{Concepts}
We need to put concepts, I've found a good source from (\citeA{Kemper})

\subsection{Technology awareness}
Let's put here something related with the main concerns that leads to this emerging technology. Also, its important to be aware that the ACID properties of DB Systems should be contrasted. In terms of each one of its meanings.

\begin{description}
    % this ideas are collected from Kemper
  \item[Atomicity] In order to achieve atomicity, the IMDB should be able to handle the effects of unsuccessful transactions. In general terms, the problem has to explicitly with the data existent in the volatile memory, since all successful transaction are in a successful state only after been committed to the logging infrastructure. % <-- PV remember, atomicity is all or nothing.
  \item[Durability] It is clear that during a failure, the IMDB Systems has the most obvious disadvantage, (without mention the hardware based solutions like batteries and so on), so the effects of a commit must be restored on a failure, this is the principle of Durability of course. One of the ideas to accomplish this is use \emph{ redo Logging }. %<-- PV important, should be more, worth to investigate 
\end{description}


% New content added from the new task splitted between our team mates:
% - My part begins with Indexing, this is not part of the data organization??, yes so I shall start with data organization.
% - Actually it is, because we have organized the structure as this document know explains, follow this document structure please.
\section{Data Organization}

Relational Databases have been the backbone of business applications for more    than 20 years, trying to provide companies with a management information system for a set of applications. During all this time, we have dreamed with the possibility of having all the information at our fingertip, we even have sold this idea\cite{Plattner}.

    But due to many reasons, we have not been able to offer this. And our systems have been separated in two main different groups: The online transaction processing (OLTP) and on line analytical processing (OLAP).

\subsection{OLTP and OLAP}
    % from \cite{Kemper}
    Under the OLTP classification, could be grouped systems with a high rate of mission-critical transactions, most of the database systems (as we know them) are mainly used for transaction processing. Some examples of OLTP systems are financial, sales, Manufacturing, order entry, banking transaction processing, human resources. All this systems performs relatively well mainly because they work on a small portion of the data. The TCP-C benchmark publish an average processing of 100.000 such sales transactions per second on a powerful system \cite{Kemper}.
    
    On the other hand, analytical, business intelligence and financial planning application were moved out to separated systems (for more flexibility and better performance). The OLAP Systems could be aimed toward specific task related to the company data warehousing. 
    
    Even though, this solutions are different in context, both are still based on \emph{ Relational theory } but with different technical approaches. 
    Moreover, recent development in the field of OLAP and the increased availability of main memory (sufficient enough to hold a completed compressed database) have enabled the processing of complex analytical requests in a fraction of a second and thus ease the development of new business processes and applications. The next step seems obviously to undo the separation between OLTP and OLAP and all requests be handled on a combined data set.

    %TODO: should I keep adding more info here? we can still talk about OLAP cubes, etc

    Now that the main memory is abundant, we have gone back to see the possibility of having a all-in-one MMDB. SAP\footnote{SAP AG: A German multinational software corporation that makes enterprise software to manage business operations and customer relations.} is one of the most enthusiastic companies pushing the use of Main Memory Databases for a mixed OLTP \& OLAP environment \cite{Plattner}, the company advertises SAP HANA as a generic MMDB solutions. 
    
    The MMDB Structure For these, the Main Memory Database Systems improve two critical points: Indexing and Storage

    
\subsection{Indexing}
Although, the performance observed in a MMDB could be outstanding compared to a on disk DB, the index structure is a critical bottleneck \cite{leisadaptive}. 

This has been copy pasted , reframe: One approach to achieving high performance in a database management system is to store the database in main memorv rather than on disk. -One can then design new data structures aid algorithms oriented towards making eflicient use of CPU cycles and memory space rather than minimizing disk accesses and  disk space efficiently \cite{lehman1986study}

``Disk vs Main Memory'' : 

Disk oriented: minimize disk access, and miminize space
Main Memory oriented: since there is no disk access -> first concern is to reduce overall computation time, and use as less memory as possible.


% Data Organization:
% How to structure a IMDB
% - Although a In Memory Database System could be arranged in different ways, one of the most important and well know is the T-Tree. T-Tree is named after its shape, lets put a big``T'' and a tree in this part or a tree with trees with a close-up showing a T shape  

\subsection{The OLTP and OLAP Systems}
% PV TODO: I think that we need to put something from OLTP and OLAP, there is something related in in Plattner:
The transactional processing part of a system OLTP (Online Transaction Processing) defines a Database system oriented to the day to day work. Almost all systems (Financial, Manufacturing, Sales, etc) rely on transactions to perform their goals. On the other hand, systems for decision taking, analytical and planning in general has been isolated from OLTPs toward specialized systems called OLAP (Online Analytical Processing).

The main purposes of this separation could be generalized into \emph{ performance } and maybe \emph{technical} issues. Even though both systems keep the essence in terms of relational theory, there are some important differences between them \cite{Plattner}.

In the IMDB Systems 

\subsection{IMDB Data Organization}
\subsubsection{Column-Store Database System}
Is a good idea to do this for IMDB?. I have some articles in \cite{ColumnOriented} and in \cite{ColumnOriented1}.

From \cite{Plattner}, it seems that although more memory has been always useful, the database systems for OLTP where not well adopted in parallel environments, one of the reasons of this were problems like deadlocks and temporary locking in parallel transactions. So in general terms has not been a good idea. Really? review this since Plattner is one of the pro Column Oriented store.

Also from \cite{Plattner}, initial tests regarding in-memory databases using relational type based on \emph{ row storage } has not shown notable performance over RDBMSs. The opportunity for \emph{column based storage} now raises due the abundant availability of main memory. 

Column-oriented databases are based on vertical partitioning by columns, this is not a new idea, Column-oriented store where considered around 70s. The core of the functioning is based in the fact that each column is stored separately, having the logical scheme built around unique positioning keys \cite{krueger2011main}. This vertical data organization offers particular advantages to reading fewer attributes, because during requests no unnecessary columns must be read, this is normally the case in row-oriented structures on hard drives.

Compression is another characteristic that Column-oriented ease. The redundancy of column store and the homogeneous domain is an advantage and a convenience for compression techniques. All the data within a column belongs to the same type, and the entropy\footnote{The semantic similarities of data} is relatively low. 

Some of the main advantages of Column Storage worth to mention:
\begin{itemize}
    \item Column Storage performs outstanding on modern CPUS (see \cite{Plattner})
    \item The vertical compression along columns is more efficient.
  \end{itemize}

\subsection{IMDB Query Processing}
\subsection{IMDB Data Recovery}

% PV: stupid idea: the memory comsumption is high since IDBM is memory based, but, OS Systems like Unix uses as much memory as they can in order to be more efficient, after coping all the memory, they start to swap and problems like trashing could happen.
% PV: stupid idea2: In a computer, all the memory has to be used because is already payed, the resource is there so take as much memory as you can.

% This is the point about conclusions and future directions
\section{Conclusions} 
The conclusions.

We could talk about: \emph{ MonetDB }, \emph{ VoltDB }


%------------------------------------------------------------------------------------
%	BIBLIOGRAPHY
%------------------------------------------------------------------------------------
% This is the part about references

\bibliographystyle{apacite}
\bibliography{bibfile}
%------------------------------------------------------------------------------------
\end{document}
