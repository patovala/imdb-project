%%%%%%%%%%%%%%%%%%%%%%%%%%%%%%%%%%%%%%%%%
% Simple Sectioned Essay Template
% LaTeX Template
%
% This template has been downloaded from:
% http://www.latextemplates.com
%
% Modified by : Ivan Patricio Valarezo (c) patovala@pupilabox.net.ec
%               The University of Melbourne - MSSE
%
%%%%%%%%%%%%%%%%%%%%%%%%%%%%%%%%%%%%%%%%%
% Why don't put the commands to compile this??
% PV: To compile the bibfile: bibtex COMP90050, then use the command \ll as usual 
%     from the vim environment
% 
%------------------------------------------------------------------------------------
%	PACKAGES AND OTHER DOCUMENT CONFIGURATIONS
%------------------------------------------------------------------------------------

\documentclass[10pt]{article} % Default font size is 12pt, it can be changed here

\usepackage{geometry} % Required to change the page size to A4
\geometry{a4paper} % Set the page size to be A4 as opposed to the default US Letter

\usepackage{graphicx} % Required for including pictures

\usepackage{float} % Allows putting an [H] in \begin{figure} to specify the exact location of the figure
\usepackage{wrapfig} % Allows in-line images such as the example fish picture

\usepackage{lipsum} % Used for inserting dummy 'Lorem ipsum' text into the template

\usepackage{url} % url package

\usepackage{apacite} % citations style apa 

\usepackage{longtable} % long tables breaking on each page 

\linespread{1.2} % Line spacing

%\setlength\parindent{0pt} % Uncomment to remove all indentation from paragraphs

\graphicspath{{./pictures/}} % Specifies the directory where pictures are stored

\begin{document}

%------------------------------------------------------------------------------------
%	TITLE PAGE
%------------------------------------------------------------------------------------

\begin{titlepage}

\newcommand{\HRule}{\rule{\linewidth}{0.5mm}} % Defines a new command for the horizontal lines, change thickness here

\center % Center everything on the page

\textsc{\LARGE University of Melbourne}\\[1.5cm] % Name of your university/college
\textsc{\Large Advanced Database Systems}\\[0.5cm] % Major heading such as course name
\textsc{\large COMP90050}\\[0.5cm] % Minor heading such as course title

\HRule \\[0.4cm]
{ \huge \bfseries In Memory Database Management Systems}\\[0.4cm] % Title of your document
\HRule \\[1.5cm]

\begin{minipage}{0.4\textwidth}
\begin{flushleft} \large
\emph{Authors:}\\
Ivan Patricio \textsc{Valarezo} % Your name
\end{flushleft}
\end{minipage}
~
\begin{minipage}{0.4\textwidth}
\begin{flushright} \large
%\emph{id:} \\ % I want to put my id here
ID: \textsc{601099} % 
\end{flushright}
\end{minipage}\\[4cm]

{\large \today}\\[3cm] % Date, change the \today to a set date if you want to be precise

%\includegraphics{Logo}\\[1cm] % Include a department/university logo - this will require the graphicx package

\vfill % Fill the rest of the page with whitespace

\end{titlepage}


%------------------------------------------------------------------------------------
%	MY PERSONAL REFERENCE SECTION
%------------------------------------------------------------------------------------
% Pato, put here all the things, tips, references, etc, that you think you'll use in 
% the future.
%
% how to cite according to the UNIMELB http://www.library.unimelb.edu.au/cite/


%------------------------------------------------------------------------------------
%	TABLE OF CONTENTS
%------------------------------------------------------------------------------------

\newpage % Begins the essay on a new page instead of on the same page as the table of contents 

%------------------------------------------------------------------------------------
%	INTRODUCTION
%------------------------------------------------------------------------------------
\begin{abstract}

Here we need to put the abstract of the work done.
 
\end{abstract}


\section{Introduction} 

The main Introduction.

%------------------------------------------------------------------------------------
%	Background and Analysis <-- TODO: is the name for this section accurate? 
%------------------------------------------------------------------------------------
% Let's talk about the main topic in terms of three major parts:
% - Data Organization
% - Query Processing
% - Data Recovery

\section{Background and Analysis} 
\subsection{Concepts}
We need to put concepts, I've found a good source from (\citeA{Kemper})

\subsection{Technology awareness}
Let's put here something related with the main concerns that leads to this emerging technology. Also, its important to be aware that the ACID properties of DB Systems should be contrasted. In terms of each one of its meanings.

\begin{description}
    % this ideas are collected from Kemper
  \item[Atomicity] In order to achieve atomicity, the IMDB should be able to handle the effects of unsuccessful transactions. In general terms, the problem has to explicitly with the data existent in the volatile memory, since all successful transaction are in a successful state only after been committed to the logging infrastructure. % <-- PV remember, atomicity is all or nothing.
  \item[Durability] It is clear that during a failure, the IMDB Systems has the most obvious disadvantage, (without mention the hardware based solutions like batteries and so on), so the effects of a commit must be restored on a failure, this is the principle of Durability of course. One of the ideas to accomplish this is use \emph{ redo Logging }. %<-- PV important, should be more, worth to investigate 
\end{description}

\subsection{IMDB Data Organization}
\subsubsection{Column-Store Database System}
Is a good idea to do this for IMDB?. I have some articles in \cite{ColumnOriented} and in \cite{ColumnOriented1}
\subsection{IMDB Query Processing}
\subsection{IMDB Data Recovery}

% PV: stupid idea: the memory comsumption is high since IDBM is memory based, but, OS Systems like Unix uses as much memory as they can in order to be more efficient, after coping all the memory, they start to swap and problems like trashing could happen.
% PV: stupid idea2: In a computer, all the memory has to be used because is already payed, the resource is there so take as much memory as you can.

% This is the point about conclusions and future directions
\section{Conclusions} 
The conclusions.

We could talk about: \emph{ MonetDB }, \emph{ VoltDB }


%------------------------------------------------------------------------------------
%	BIBLIOGRAPHY
%------------------------------------------------------------------------------------
% This is the part about references

\bibliographystyle{apacite}
\bibliography{bibfile}
%------------------------------------------------------------------------------------
\end{document}
